% Why do I want to work at AIND?
    % They are accelerating the pace of scientific progress
    % The system they are building allows SWEs the chance to help address the growing reproducibility crisis in science

% Why should AIND want me to work for them?
    % 1a.) I would be good at building a tool for collecting and managing large-scale data
    % 1b.) I have experience building visualizations and reporting tools that monitor the quality of incoming data
    % 2.) I have experience with cluster and cloud based computing on large scale data
    % 3a.) I can collaborate with teams of scientists and engineers to develop scientific software. 
    % 3b.) I have experience with written and verbal communication skills in a scientific setting

I am a software engineer with an MS in Computer Science and four years of experience developing software for data-intensive systems in both academic labs and industry.
As a software engineer who is passionate about accelerating the pace of scientific advancement and ameliorating the growing reproducibility crisis, I am extremely enthusiastic to be applying for the position of Software Engineer II at the Allen Institute for Neural Dynamics.

During my time in the ML4AI Laboratory as well as at Rocket Mortgage I built several software systems that consisted of a machine learning model processing on top of a terabyte-scale data source.
As part of this process, I quickly learned about the necessity of monitoring data sources to ensure the quality of input data into the ML models remained high as well as to detect anomalies when they inevitably occur.
I believe my expertise in creating data monitoring and anomaly detection systems could lend to AIND's goal of ensuring high quality data on its scientific data platform.

Working as a software engineer in both an academic and industry setting has allowed me to adapt my expertise to diverse goals.
I have gained experience utilizing a wide array of distributed computing resources while adhering to the best practices of software development.
Additionally, I have written software programs for both CPU-intensive and GPU-intensive tasks that leverage high-performance computing clusters to meet the demands of supersized datasets.
Further, I have designed complex software systems that employ cutting edge cloud computing technologies in order to ensure they remain highly available, fault tolerant, and able to scale as needed.
This combination of experiences provides me with the skills necessary to successfully develop robust data pipelines that accelerate the collection, processing, and analysis of large-scale neurophysiology data.

My varied array of experiences have built up my  technical expertise, and I am well-suited to apply this expertise to important scientific missions of today and the future.
Throughout my career, I have greatly enjoyed closely collaborating with scientists by communicating solutions to complex computation problems and brainstorming novel approaches to answer questions in the fields of agricultural science, planetary science, and microbiology.
Throughout these experiences, I have met challenges head-on by being continually eager to learn new skills inside and outside of my comfort zone.
Undoubtedly, the most exciting aspect of the opportunity to work as a software engineer at the AIND would be having the privilege to collaborate closely with experts in the field of neurobiology.
                        
Having the opportunity to apply my talents at the Allen Institute will provide me with a sense of fulfillment in knowing I will play a critical role in the quest to discover how our minds make sense of the complex world around us.
It would be an honor to interview for this position, and I look forward to hearing from you.
Thank you for your consideration. 
    