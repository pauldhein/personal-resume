%------------------------------------------------------------------------------
%	SECTION TITLE
%------------------------------------------------------------------------------
\cvsection{Experience}

\vspace{-0.2cm}
%------------------------------------------------------------------------------
%	CONTENT
%------------------------------------------------------------------------------
\begin{cventries}
  %---------------------------------------------------------
  \cventry
    {Senior Machine Learning Engineer} % Bottom-left
    {Rocket Mortgage} % Top-left (bold)
    {Sept. 2021 --- Aug. 2023} % top-right (highlight color)
    {June 2023 --- Aug. 2023} % Bottom-right
    {
      \vspace{-0.1cm}
      \begin{cvitems} % Description(s) of tasks/responsibilities
        % \item Improved a lead delivery service by eliminating unreported errors and reducing remaining errors by 90\% using AWS Lambda and SQS message timers.
        % \item Lead a compute cost reduction of up to 95\% for several data pipelines by training colleagues to use Apache Spark to improve data pipeline efficiency.
        \item Automated ETL pipeline validation by creating a dataset synthesizer RESTful service using Synthetic Data Vault, \textbf{FastAPI}, \textbf{Docker}, and \textbf{Kubernetes}.
        \item Created a monitoring system with \textbf{AWS Lambda}, \textbf{AWS SQS}, and \textbf{AWS CloudWatch} that eliminated existing errors for a marketing lead delivery service.
        \item Lead a compute cost reduction of up to 95\% for several pipelines by leveraging \textbf{Apache Spark} and data persistence to improve data pipeline efficiency.
      \end{cvitems}
    }
  \vspace{-.2cm}

  \cventry
    {Machine Learning Engineer} % Bottom-left
    {} % Top-left (bold)
    {} % top-right (highlight color)
    {Sept. 2021 --- May 2023} % Bottom-right
    {
      \vspace{-0.1cm}
      \begin{cvitems} % Description(s) of tasks/responsibilities
        \item Improved the data throughput of a marketing attribution model using \textbf{Apache Spark} and \textbf{AWS EMR} to allow terabyte-scale data to be processed daily.
        \item Improved the technical maturity of a junior engineer through \textbf{mentorship} and \textbf{pair-programming} which lead to them receiving a promotion.
        \item Translated a proof-of-concept bayesian model for paid search optimization from \textbf{R} into \textbf{Python} using \textbf{Pandas}, \textbf{awswrangler}, \textbf{NumPy}, and \textbf{PyMC3}.
        \item Deployed a paid search optimization model to an \textbf{AWS SageMaker} endpoint capable auto-tuning Google Ads keyword bids with real-time inference.
        \item Created a development + deployment environment using \textbf{Bash}, \textbf{CircleCI}, and \textbf{Terraform} that reduced ML model rollout time from days to hours.
        % \item Reduced ML model update rollout time from days to hours by aligning the data scientist experiment environment with the production environment.
      \end{cvitems}
    }
  \vspace{.2cm}
%---------------------------------------------------------
  \cventry
    {Research Software Engineer} % Bottom-left
    {ML4AI Laboratory} % Top-left (bold)
    {June 2016 --- Sept. 2021} % top-right (highlight color)
    {June 2019 --- Sept. 2021} % Bottom-right
    {
      \vspace{-0.1cm}
      \begin{cvitems} % Description(s) of tasks/responsibilities
        % \item Contributed to the lab being awarded a new DARPA research grant by designing an ML model for generating Python source code from assembly code.
        \item Implemented a \textbf{Naïve Bayes} model, a \textbf{Bi-LSTM}, and a deep \textbf{CNN} using \textbf{PyTorch} for classifying biological taxonomy from DNA sequences.
        \item Designed an \textbf{encoder-decoder model} for generating Python  code from assembly code that led to the lab being awarded a DARPA research grant.
        % \item Implemented a source code dataflow graph interchange format that became the standard for all external collaborators on a DARPA research grant.
        \item Utilized the \textbf{PyTorch} DataParallel module, and \textbf{Slurm} to achieve a 6x training acceleration for a sequence translation network on a GPU cluster.
        % \item Achieved 6x training acceleration for sequence translation neural nets by implementing data parallelism in PyTorch over a distributed 8 GPU cluster.
        \item Provided graduate students with \textbf{technical mentorship} on using scientific Python libraries and Docker to conduct reproducible ML experiments.
      \end{cvitems}
    }
    \vspace{.07cm}
%---------------------------------------------------------
  \cventry
    {Graduate / Undregraduate Research Assistant} % Bottom-left
    {} % Top-left (bold)
    {} % top-right (highlight color)
    {June 2016 --- June 2019} % Bottom-right
    {
      \vspace{-0.1cm}
      \begin{cvitems} % Description(s) of tasks/responsibilities
        \item Implemented \textbf{feature selection} and \textbf{class imbalance} correction routines for a relation extraction model leading to a 45\% improvement in precision.
        % \item Improved the precision of an inter-sentence biomedical relation extraction model by 26 points using rigorous feature engineering and model selection.
        \item Designed a parallel hyperparameter grid search program in \textbf{Python} capable of tuning any \textbf{Scikit-learn} classifier on a distributed computing cluster.
        \item Created a corpora of musical patterns from jazz solos using a \textbf{spatial pattern discovery algorithm} for training an ML jazz solo generation model.
        \item Created a web application with \textbf{Python}, \textbf{Flask}, and \textbf{D3.js} capable of allowing an AI jazz generation model to record duets with a human musician.
      \end{cvitems}
    }
    \vspace{.07cm}
%---------------------------------------------------------
  % \cventry
  %   {Undergraduate Research Assistant} % Bottom-left
  %   {} % Top-left (bold)
  %   {} % top-right (highlight color)
  %   {June 2016 --- May 2017} % Bottom-right
  %   {
  %     \vspace{-0.1cm}
  %     \begin{cvitems} % Description(s) of tasks/responsibilities
  %       \item Created a corpora of musical patterns from jazz solos using a \textbf{spatial pattern discovery algorithm} for training an ML jazz solo generation model.
  %       \item Created a web application with \textbf{Python}, \textbf{Flask}, and \textbf{D3.js} capable of allowing an AI jazz generation model to record duets with a human musician.
  %     \end{cvitems}
  %   }
  %   \vspace{.2cm}
%---------------------------------------------------------
  \cventry
    {Student Programmer} % Bottom-left
    {Lunar Planetary Laboratory} % Top-left (bold)
    {April 2015 --- June 2016} % top-right (highlight color)
    {} % Bottom-right
    {
      \vspace{-0.1cm}
      \begin{cvitems} % Description(s) of tasks/responsibilities
        \item Assisted in developing a web application using \textbf{Node.js} that enabled scientists across the globe to view, create, and catalog spacecraft telemetry data.
        \item Assisted in designing a \textbf{database ERD} and implementing a \textbf{SQL schema} for pedigree tracking of data products originating from telemetry data.
      \end{cvitems}
    }
%---------------------------------------------------------

\end{cventries}
